
%%%%%%%%% PROPOSAL -- 15 pages (including Prior NSF Support)

\section*{\Large Organizational Responsiveness to External Demands:  A Modeling Approach based on Statistical Text and Network Analysis}


% From the NSF Grants Proposal Guide:
% "The Project Description should provide a clear statement of the work 
% to be undertaken and must include: objectives for the period of the proposed 
% work and expected significance; relation to longer-term goals of the PI's 
% project; and relation to the present state of knowledge in the field, 
% to work in progress by the PI under other support and to work in progress 
% elsewhere."

\section{Introduction}

Nearly every organization strives to respond in a timely and accurate manner to the needs and demands of some external constituency. Firms respond to customers, governments respond to citizens and educational institutions respond to students. The rapid advancement in communications technology over the last two decades has forever transformed the nature, volume and sources of input and feedback available to organizations. Also, electronic communications have drastically improved the ability of organizations to document and communicate their internal developments. These complimentary developments have had a transformative impact on governance - moving to what {\bf CITE} call 'we government'. Most elected officials can be directly contacted electronically through simple internet pools. Citizens can advertise and sign petitions on the web and attend internet 'town meetings' with their representatives. Regarding the internal activities of government; citizens can access electronic communications of their officials through public records requests, access meeting minutes on the web and, e.g., watch the floor activities of the US House of Representatives on HouseLive.gov. 

In this project we will develop and apply methods for identifying the cycle of input, response and feedback that leaves its fingerprint in the electronic communications record. We will focus on the nexus between government organizations and their constituents, but the methods we develop will be portable to other types of organizations. Government responsiveness to citizen input offers an ideal venue within which to model the relationship between streams of textual records. First, in democratic societies there is a common expectation that the government will respond to public demands. Second, most of the input modes on which we will focused were designed precisely for the objective of providing input to which public officials could respond. Third, and perhaps of greatest practical importance, due to the scope of freedom of information laws in the US, we as researchers can access the public input and internal communications data associated with a multitude of government organizations.

We frame this project by associating different phases in the cycle of governance with four different types of textual streams - public input (e.g., emails from citizens to government officials, informal internal communications (e.g., emails among officials), formal deliberations (e.g., legislative meeting minutes) and policy outputs (e.g., regulations, laws). We seek to understand these textual themes through the lens of statistical topic models {\bf CITES}. We will develop and apply models that permit the identification of the ways in which topics rise and fall within domains and, crucially, are related across domains. The result will be an analytical approach that permits an organization to distill and investigate the dynamics of input, responsiveness and feedback through a common framework of statistical text analysis.  The methods we develop will offer answers regarding several pertinent questions about organizational management of outside input, e.g., is organizational attention to a topic proportional to its attention in outside input, how does an organization adapt to the rise of issues that are novel relative to its current foci, is responsiveness timely? 

Topic models infer discrete topics from a corpus of documents. A topic is simply a relative frequency distribution of words and each document is probabilistically associated with each topic identified {\bf CITES}. Statistical topic models provide a dually qualitative and quantitative inferential summary of textual corpora. Qualitative in that the textual content of a corpus is maintained and words themselves form the basis of the quantitative analysis.  Dynamic topic models provide an excellent framework within which to understand input to, output from and feedback to organizations that document their activities at various stages in a textual format. Since the seminal work on statistical topic models {\bf CITE}, the basic framework has been extended and adapted to focus on several aspects of textual corpora {\bf CITE}, including author-specific attributes of text {\bf CITE}, dyadic (i.e., author-recipient) aspects of messages {\bf CITE}, dynamics, the underlying communication network, and joint text-metadata models of documents {\bf CITE}. In the current project, we will undertake an ambitious set of extensions that integrate several of these extensions - jointly modeling separate streams of text that influence each other, are informed by rich meta-data, incorporate the underlying communication network, and characterize the over-time aspect of the text streams.

{\bf Figure 1} Illustrates the cycle of organizational responsiveness that we intend to model through the guise of co-evolving textual streams. Considering the case of governance, substantial research exists that focuses on parts of this cycle. For example, a large body of research exists that documents recent developments in tools for citizens to provide precise, timely and voluminous input to government officials {\bf CITES}. There is also a large body of research focusing on legislative adaptation to broad ideological trends among constituents {\bf CITES}.  And, yet another literature that addresses the processes by which topics rise from informal awareness among officials and outside parties to the legislative agenda {\bf CITES}.. However, due to the historical inaccessibility of timely and common data modes related to each component of the governance cycle, little research has endeavored to connect all of the dots. We will provide such a complete picture, leveraging the common, available, and timely mode of text streams.

\section{Motivating Example}

A couple paragraphs on an external-input system.

\section{Problem Statement}

Here we distill problem to abstract structure.

\section{What has been done so far}

Here we describe the pilot work, publication and data collection, as well as related research.

\section{Proposed New Work}

Here we describe the methods we intend to develop, datasets we intend to create and applications we intend to develop.

\section{Timeline and Division of Labor}

Here we give a clear breakdown of how the work will be divided and what will be completed.

\section{Broader Impact}

Here we discuss broader scientific impact, training and education, as well as potential non-academic impacts.

\section{Results From Prior NSF Support}

% 5 pages or fewer of the 15 pages for entire description document.
% include results from NSF grants received in the past 5 years.
% if supported by more than one grant, choose the most relevant one
% for each grant, include: NSF award number, amount, dates of
% support, and publications resulting from this research.
% due to space limitations, it is often advisable to use citations rather
% than putting the titles of the publications in the body 
% of this section

% e.g.: "My prior grant, "Uses of Coffee in Mathematical Research" (DMS-0123456, 
% $100,000, 2005-2008), resulted in 3 papers [1],[2],[3], demonstrating..."

% if requesting postdoctoral research salary, a supplemental 1-page document
% called "Postdoc Mentoring Plan" will be required 

