
%%%%%%%%% SUMMARY -- 1 page, third person
% e.g:  "The PI will prove" not "I will prove"

% big set-up, project description
\noindent {\bf Project Summary}

\noindent The electronic exchange of text (e.g., e-mail, text messaging, micro-blogging) now constitutes the cornerstone of the communication process in most organizations. Also, many organizations archive their communications for record-keeping purposes. Substantial recent research shows how important the patterns and structure of communication, formalized as communication networks, are to effective organizational and individual problem-solving. Domains studied include emergency responses to natural disasters, the dissemination of preventive health behaviors, national defense and problem-solving in  behavioral experiments. Archives of textual communications offer substantial promise for measuring, diagnosing and optimizing intra-organizational communication networks. However, several challenges stand in the way of realizing this promise. Textual archives constitute complex, large  and multi-scale data sources stored in varied and non-standard formats. On the machine learning end, effectively leveraging this data will require scalable algorithms that effectively organize and exploit the content, contextual, temporal and threading information associated with textual communications. From the social scientific perspective, effective methods will offer results that are insightful and interpretable to scholars of social organizations and practitioners alike. \vspace{.25cm}

% Technical details, building credibility

\noindent {\bf Methodology:} We propose to study the ways in which organizations' email corpora can be analyzed to improve intra-organizational communication networks. We will use North Carolina and Florida county government email archives acquired via public records requests and online data collection. We will design computational tools that incorporate (1) content, (2) the communication ties and (3) additional metadata such as temporal and threading information.  The methods that we develop will be capable of insightful visualizations that permit the intuitive exploration of the content and context of intra-organizational communication networks, and will be anchored in realistically complex probabilistic models of real-world communication processes.  The data we collect will permit extensive validation and innovative application of the algorithms we develop. We will relate the core email data with additional publicly available data on county governments, including regulations/legislation and minutes from county legislatures. The proposed research will be conducted by an interdisciplinary team that brings expertise in the computational (Wallach) and social scientific (Desmarais) aspects of textual communication network analysis.\vspace{.25cm}

% Differentiating our approach from others (intellectual merit)

\noindent {\bf Intellectual Merit:} This project will offer important contributions to both computational and social sciences. In terms of computational approaches, we will enhance methods for the statistical analysis of text and network data, including (a) the extension of a multinetwork spatial embedding model to leverage multiple node sets and temporal information, (b) develop scalable inference to capitalize on massive communication archives, and (c) model content-flow dynamics within and across textual modes.  On the social science side, the methods we develop and data we collect will advance our understanding of government organizations. By precisely measuring dozens of government communication networks, we will offer an unprecedented assessment of the common structures that characterize intra-organizational communication patterns. Additionally, by integrating (1) communications in public fora, (2) intra-governmental communications, and (3) legislative activities, we will offer a novel micro-level characterization of the local government policymaking cycle. \vspace{.05cm}

% Broader Impa

\noindent {\bf Broader Impact:} This project will provide essential tools for organizations maximize benefits of their internal communication patterns. This holds potential to, e.g., improve organizations' management of natural disasters, assist officials in coordinating to maintain national security, and aid in designing firms to usher in technological innovation.




