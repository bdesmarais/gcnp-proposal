
\required{Data Management Plan}


\subsection*{A. Project Information}
This Data Management Plan (DMP) covers the data which will be collected for a the ERGM Tools project at the University of Massachusetts at Amherst (UMASS-Amherst) and the University of North Carolina at Chapel Hill (UNC-CH). The study projected to be conducted between January 1, 2013 and December 31, 2014. The study will collect non-sensitive data, most of which is already archived in free internet datasets, about countries, legislators, other social units, and the relationships among them. Because all of the information we would compile is non-classified and, already, open source, we do not anticipate any confidentiality concerns. 

Any and all data we use, we will consolidate from existing sources, which are already freely available to the general public. Data consolidation should be exceedingly simple because it is, essentially, a matter of merging and reformatting the contents of several freely available datasets on related topics. The data collected during this study will be archived on PI Desmarais' workstation in the Department of Political Science at UMASS-Amherst. The data will be stored in a specific virtual archive and will be made publicly available through the web pages and Dataverse Network (DVN) pages of both PIs.


\subsection*{B. General Data Management Plan Information }
This DMP was created on July 27, 2012 for submission to the National Science Foundation as
required by NSF guidelines in the interest of securing funding for this study. The aim and purpose of this DMP is to detail and guarantee the preservation of the data collected during this study, as well as any results derived from the associated research. This DMP is intended for review by relevant NSF
personnel, as well as UMASS-Amherst and UNC-CH staff affiliated directly with this study and the collection and preservation of the associated data and research. This is the first iteration of this DMP associated with this data.

\subsection*{C. Policies }
There are no requirements stipulated by the funding or partner organizations regarding this data.
Comprehensive institutional and research group guidelines specified by UNC-CH were applied regarding the collection of this data. There are no additional requirements associated with the data being submitted.

\subsection*{D. Legal Guidelines and Requirements }
This study will only collect non-sensitive data. No personal identifiers are possible as all observations are on the level of publicly visible attributes of countries, public (political) figures, or the relationships between them. There are no copyright or licensing issues associated with the data being submitted.

\subsection*{E. Access, Sharing and Re-use of Data }

The researchers associated with this study are not aware of any reasons which might prohibit the
sharing and re-use of the data being submitted. The researchers are not required to make this data
available publicly but have elected to do so. The data being submitted will be made publicly available
through the web pages and Dataverse Network (DVN) pages of both PIs by September 1, 2016. There
will be no additional restrictions or permissions required for accessing the data. %Findings will be
% published by the researchers based on this data; the estimated date of publication is September 1, 2015. There is an agreement regarding the right of the original data collector, creator or principal investigator for first use of the data. The specified embargo period associated with the data being submitted extends from the projected conclusion date for initial research (September 1, 2016) until six months after projected publication date for the findings (September 1, 2016). The embargo will be lifted by March 1, 2017.

\subsection*{F. Data Standards and Capture }
The data will be captured using Microsoft Excel and analyzed using R data analytics tools. The researchers are not aware of any issues regarding the effects or limitations of these formats regarding the data being submitted.


\subsection*{G. Security, Storage, Management and Back-Up of Data }

The PI's experience with, and commitment to, secure data archiving is well established and
is in keeping with established UNC Information Security Policies. During the implementation of the
study, associated research data will be physically stored on a password-protected secure server
maintained by PI Desmarais in the Political Science department at UMASS-Amherst using standard R file formats. %Though no sensitive data will be collected in the course of this study, no data will reside on portable or laptop devices, and no other external media/format(s) will be used for data storage.

Research data is backed up on a daily basis. The researchers are currently responsible for storage,
maintenance and back-up of the data. The specific storage volume of the data being submitted will be
not more than 1GB maximum. The long-term strategy for the maintenance, curation and archiving of
the data will be implemented when the data and associated research are migrated to the DVN.

\subsection*{H. Preservation, Review and Long-Term Management of Data }
Data collected during this study will be archived with the Dataverse Network (DVN). The data will be
stored in a specific virtual archive and will be made publicly available through the DVN.
As a result of this arrangement, there are no specific financial considerations of which the researchers
are currently aware that might impact the long-term management of the data. 







